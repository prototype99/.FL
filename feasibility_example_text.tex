\documentclass[12pt]{article}
\usepackage{a4}
\usepackage{times}
\usepackage{graphicx}
\usepackage[style=authoryear,autocite=inline,backend=biber]{biblatex}
\bibliography{feasibility_example}

\title{Feasibility Study}
\date{}
\begin{document}
\maketitle
\Large
\begin{tabular}{ll}
Student name: & \\
Student number: & \\
Degree course: & \\
Supervisor name: & \\
Project title: & \\	
\end{tabular}

\small
\subsection*{Course-Specific Learning Outcomes}

\subsection*{Project Background}
	
	Just a few words so you can see how the citation system works.
	
	The idea or definition of a computer game, and their relationship to other games, is complex but can be determined \autocite{juul-2003}.
	
	\textcite{juul-2003} determines the nature of computer games and their relationship with other game-types. 
	
	
\subsection*{Aim}
\subsection*{Objectives}
\subsection*{Problems}
\subsection*{Required Resources}
\subsection*{Schedule}
\subsection*{Ethics number}
	








\printbibliography
\end{document}
